\documentclass[11pt,english,a4paper]{article}
\usepackage{babel}
\input{/home/marius/Dokumenter/preamples/phys_en.pre}
\author{\normalsize Marius Jonsson (Institutt for Vanskelig Fysikk, Oscars gate 19, 0352 OSLO, Norway) \\\\
\vspace{5px}
\normalsize \texttt{http://github.com/kingoslo/flintstones}}
\title{\bf \uppercase{Some title}}
\date{\normalsize \today}
\addbibresource{/home/marius/Dokumenter/MyLibrary.bib}
\DeclareUnicodeCharacter{2212}{$-$}
\begin{document}
\maketitle
\begin{abstract} \normalsize This is a report submission for the first project of «Computational physics 2» at the Institute of Physics, University of Oslo, autumn 2016.
\end{abstract}
\lstset{
  xleftmargin=.2\textwidth, xrightmargin=.2\textwidth
}

\section*{\uppercase{Introduction}}
A.\\
\\
The report is structured by «introduction»-, «methods»-, «results and discussion»- and finally a «conclusion and perspectives»-sections.
\section*{\uppercase{Methods}}
$\varrho \varsigma \vartheta \varpi$
\section*{\uppercase{Results and discussion}}

\section*{\uppercase{Conclusion and perspectives}}

\section*{\uppercase{Appendix}}
\begin{theorem}[Hastings-Metropolis theorem]
Suppose that $C\pi_i$ is a discrete probability distribution. If $q_{ij}$ is any irreducible transition probability matrix, and $X_n$ is a Markov chain with transition probability matrix 
\begin{equation}
P_{ij} = \ccases {
\alpha_{  ij} q_{  ij} , \quad &j\neq i\\
q_{  ii} + \sum_{k=0}^\infty q_{  ik}(1 - \alpha_{  ik})\quad & j=i
},\qquad \text{where} \qquad \alpha_{  ij} = \min \left( \frac{\pi_j q_{  ji}}{\pi_i q_{  ij}} , 1\right), \label{eq:metropolis}
\end{equation}
\end{theorem}
then $X_n$ is time reversible with stationary distribution $\pi_i$.
\begin{proof}
Assume that the hypothesis is true. Then in particular $q_{  ij} \neq 0$ for all $i,j$ since $q$ is irreducible. Notice that if 
\[
\frac{\pi_j q_{  ji}}{\pi_i q_{  ij}} = 1,
\]
then there is nothing to prove since then $\alpha_{ij} = 1$ and $\alpha_{ji} = 1$, and therefore
\begin{equation}
\pi_iP_{  ij} = \pi_jP_{  ji} \label{eq:detailed_balance}
\end{equation} 
is automatic. Hence it suffices to prove \eqref{eq:detailed_balance} for the two cases
\[
\frac{\pi_j q_{  ji}}{\pi_i q_{  ij}} > 1 \qquad \text{and} \qquad \frac{\pi_j q_{  ji}}{\pi_i q_{  ij}} < 1,
\]
separately. Suppose first that $\pi_j q_{  ji} > \pi_i q_{  ij}$ $(\dagger)$. Write:
\[
\pi_iP_{  ij} \stackrel{\eqref{eq:metropolis}}{=} \pi_i  q_{  ij} \alpha_{  ij} \stackrel{\eqref{eq:metropolis}(\dagger)}{=} \pi_i q_{  ij} \cdot 1 =
\pi_i q_{  ij}  \frac{\alpha_{  ji}}{\alpha_{  ji}} = \alpha_{  ji}\pi_i q_{  ij}  \frac{1}{\alpha_{  ji}} \stackrel{(\dagger)}{=} \alpha_{  ji}\pi_i q_{  ij}  \frac{\pi_j q_{  ji}}{\pi_i q_{  ij}} = \alpha_{  ji}  \pi_j q_{  ji} \stackrel{\eqref{eq:metropolis}}{=} \pi_jP_{  ji}. 
\]
In the case that $\pi_j q_{  ji} < \pi_i q_{  ij}$ $(\ddagger)$, write
\[
\pi_iP_{  ij} \stackrel{\eqref{eq:metropolis}}{=} \pi_i  q_{  ij} \alpha_{  ij} \stackrel{\eqref{eq:metropolis}(\ddagger)}{=} \pi_i q_{  ij} \frac{\pi_j q_{  ji}}{\pi_i q_{  ij}} = \pi_j q_{  ji} = \pi_j q_{  ji} \cdot 1 \stackrel{\eqref{eq:metropolis}(\ddagger)}{=} \pi_j q_{  ji} \cdot \alpha_{  ji} = \pi_jP_{  ji},
\]
which means $X_n$ is time reversible with stationary probability $\pi_i$.
\end{proof}

\printbibliography
\end{document}